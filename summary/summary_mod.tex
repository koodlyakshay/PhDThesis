\chapter*{Summary}
\addcontentsline{toc}{chapter}{Summary}
\setheader{Summary}

Wind turbine aerodynamics spans a wide range of scales starting from the millimeter thick boundary layers up to atmospheric flows. Analysis of each of these scales have traditionally been performed using simplified or 'engineering' models. However, with simplifications also come uncertainties. In order to reduce the uncertainties and also to improve the design process, more accurate analysis tools are necessary. Computational Fluid Dynamics (CFD) based methods are being used to fill this need. This thesis presents the development and application of an incompressible pressure based solver. The new solver has been developed within the framework of SU2, an open source collection of C++ based software tools for multi-physics analysis. 

Wind turbines operate under "incompressible" conditions and the CFD methods are based on solving the Navier-Stokes equations in the incompressible form. The main difficulty in solving the incompressible Navier-Stokes equations lies in resolving the pressure velocity coupling. Various approaches have been proposed over the years to overcome this difficulty and in this work the SIMPLE algorithm is used. The common link among all the approaches to resolve the pressure velocity coupling was the use of staggered grids, where pressure and velocities are stored at cell centers and cell faces respectively. The staggered grid comes with its own disadvantages, especially for modern day industrial problems. A momentum based interpolation technique is used to allow for the SIMPLE algorithm to be applied on collocated grids. The effect of turbulence is modeled using the eddy viscosity based Spalart-Allmaras (SA) and the $k$-$\omega$ Shear Stress Transport (SST) models. Verification of the new solver is carried out using Couette flow problems where analytical solutions to the Navier-Stokes equations can be found. Validation of the new solver is carried out using widely studied problems like flow over a flat plate, backward facing steps and cylinder. 

CFD is used for a wide variety of problems in the wind turbine industry. Three such problems are studied in this work. First, the effect of vortex generators on the boundary layer is presented. Integral boundary layer (IBL) based methods are used extensively for quick and accurate analysis of airfoil performance. The effect of vortex generators on the boundary layer is analyzed using the mixing layer theory as a first step towards modeling vortex generators in IBL methods. Another issue gaining importance recently is the effect of leading edge erosion. As the blade surface is exposed to the elements continuously, it is prone to erosion which leads to a reduction in performance. A roughness model is implemented and used to analyze the effect of erosion on aerodynamic performance and boundary layer behavior. Finally, the new solver is used to study the flow past a rotating wind turbine blade.

\chapter*{Samenvatting}
\addcontentsline{toc}{chapter}{Samenvatting}
\setheader{Samenvatting}

{\selectlanguage{dutch}
De aerodynamica van windturbines omvat een brede diversiteit aan schalen, van de millimeter dikke grenslagen tot atmosferische stromingen. Analyse van elk van deze schalen werd traditioneel grotendeels uitgevoerd met behulp van vereenvoudigde of 'engineering' modellen. Echter, vereenvoudigingen introduceren ook onzekerheden. Om deze onzekerheden te verminderen en ook om het ontwerpproces te verbeteren, zijn nauwkeurigere analyse-methoden nodig. Om in deze behoefte te voorzien, worden methoden gebruikt, die gebaseerd zijn op Computational Fluid Dynamics (CFD). Dit proefschrift presenteert de ontwikkeling en toepassing van een onsamendrukbare, pressure based solver. De nieuwe solver is ontwikkeld binnen het raamwerk van SU2, een open source verzameling van op C++ gebaseerde softwaretools voor multi-disciplinaire analyse.

Windturbines werken onder $"$onsamendrukbare$"$ omstandigheden en de CFD methoden zijn gebaseerd op het oplossen van de Navier-Stokes vergelijkingen in de onsamendrukbare vorm. De grootste moeilijkheid bij het oplossen van de onsamendrukbare Navier-Stokes-vergelijkingen ligt in het oplossen van de koppeling tussen de druk en snelheden. In de loop der jaren zijn er verschillende benaderingen voorgesteld om deze moeilijkheid te overwinnen en in dit werk wordt het SIMPLE-algoritme gebruikt. De overeenkomst tussen alle benaderingen voor het oplossen van de koppeling tussen druk en snelheden was het gebruik van staggered roosters, waar druk en snelheden worden opgeslagen in respectievelijk het midden en op de randen van de cel. Het staggered rooster heeft zijn eigen nadelen, vooral voor moderne industri\"ele problemen. Een op impuls gebaseerde interpolatietechniek wordt gebruikt om het SIMPLE-algoritme toe te passen op collocated grids. Het effect van turbulentie is gemodelleerd met behulp van de eddy-viscositeit modellen van Spalart-Allmaras (SA) en $k$-$\omega$ Shear Stress Transport (SST). Verificatie van de nieuwe solver wordt uitgevoerd met behulp van Couette-stromingsproblemen waarvoor analytische oplossingen van de Navier-Stokes vergelijkingen gevonden kunnen worden. Validatie van de nieuwe solver wordt uitgevoerd met behulp van veel bestudeerde problemen, zoals de stroming over een vlakke plaat, backward facing step en cilinder.

CFD wordt gebruikt voor een breed scala aan problemen in de windturbine-industrie. In dit werk worden drie van deze problemen bestudeerd. Ten eerste wordt het effect van wervel-generatoren op de grenslaag gepresenteerd. Grenslaag Integraal Methoden (Engelse afkorting IBL) worden veelvuldig gebruikt voor snelle en nauwkeurige analyse van de prestaties van het vleugelprofiel. Het effect van wervel-generatoren op de grenslaag wordt geanalyseerd met behulp van de mixing layer theorie als een eerste stap naar het modelleren van wervel-generatoren in grenslaag-integraalmethoden. Een ander probleem dat recentelijk steeds belangrijker wordt, is het effect van erosie aan de voorkant van het blad. Omdat het bladoppervlak continu wordt blootgesteld aan de elementen, is het gevoelig voor erosie, wat leidt tot een vermindering van de prestaties. Een ruwheidsmodel is ge\"{\i}mplementeerd en gebruikt om het effect van erosie op de aerodynamische prestaties en het gedrag van de  grenslaag te analyseren. Ten slotte wordt de nieuwe solver gebruikt om de stroming over een draaiend windturbineblad te bestuderen.

}

