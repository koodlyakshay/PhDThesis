\section{Conclusions and future outlook}\label{sec:conclusions}
Roughness models for two RANS turbulence models, SA and SST, were implemented in SU2 and the accuracy was examined via grid refinement. The models were validated against empirical models for the shift in velocity profiles in the boundary layer and experimental skin friction data for flat plates. It was seen that that the SST roughness model required a much finer grid compared to the SA roughness model to give a grid independent solutions. However, despite the finer grid the results from the SST roughness model did not match the experimental data or the empirical models under fully rough conditions, unlike the SA roughness model. Based on these results the SA roughness model was further validated against experimental data on two different airfoils. The SA model predicted the reduction in lift for different roughness levels accurately for the NACA $65_2215$ airfoil. The SA model was also validated against an experiment with negative roughness (pits and gouges) on the DU-96-W-180 airfoil. Encouraging results were observed for both roughness levels tested. The statistical method to determine the equivalent sand grain roughness proved to be accurate. Some differences were observed in the clean simulation, most likely due to the fact the simulations were run under fully turbulent conditions, unlike the experiments. 

Further, the behavior of different integral boundary layer properties like displacement thickness, momentum thickness, shape factors and closures were investigated for the NACA $65_2215$ airfoil. The existing skin friction closure relations for clean surfaces greatly underpredict the skin friction ($C_f$) and are not valid for rough surfaces. However, the closure relation for the kinetic energy shape factor ($H_k$) performed well for low roughness levels ($Re_{k_s} < 1000)$) but deviated at higher roughness levels and under separation. The deviation was only marginal compared to the skin friction closure relation. However, since the closure relations for other quantities like the dissipation coefficient and the Reynolds shear stress coefficient depend on $C_f$ and $H_k$ new closure relations will be needed in order to simulate rough surfaces in integral boundary layer tools like RFOIL.

% Eroded turbine blades are modeled as rough surfaces. Roughness has a significant effect on flow and can lead to loss in lift and increased drag. 
The main focus of this study was on the effect of roughness on turbulent boundary layers. For laminar boundary layers, roughness leads to premature transition to turbulence.  In order to fully capture the effect of roughness, the effect on transition will also be considered in the future. Further, more boundary layer data at different roughness levels are needed to derive new closure relations for integral boundary layer methods.
%\newpage