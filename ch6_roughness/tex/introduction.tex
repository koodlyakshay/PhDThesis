% =============================================
\section{Introduction}

Leading edge erosion is an issue of growing concern in the wind turbine industry in recent years. The combination of growth in the size of wind turbines, increased offshore installations, especially in locations with more adverse weather conditions, has made this subject crucial to the industry \cite{HERRING2019109382}. Erosion of turbine blades are largely caused by rain, hailstones, accumulation of contaminants and tends to change the shape of the airfoils. This leads to a reduction in aerodynamic performance of the affected sections. Han\citep{han2018effects} presented the effects of contamination of the airfoil used at blade tips on a $5$ MW NREL turbine blade using CFD simulations. They report a worst case scenario where the Annual Energy Production (AEP) drops by $3.7\%$. Herring\cite{HERRING2019109382} presents a thorough review on the growing importance of leading edge erosion and different coating alternatives to reduce the impact of erosion. A wide range of drop in AEP, from about $25\%$ to about $3.7\%$, is reported and the authors suggest it is due to different operating conditions and roughness levels used to evaluate the impact of erosion. The authors also note that repair of moderate erosion can recover the AEP by about $2\%$. 


In order to quantify the adverse effects of roughness the flow around the turbine blades should be investigated. Laminar flow tends to transition to turbulent flow prematurely in presence of roughness. A review of experimental approaches to model roughness and its effect on transition can be found in Ehrmann et al \cite{ehrmann2017effect}. Langel et al. \cite{langel2017rans} performed experiments on two airfoils by adding cut vinyl decals and focused on $100 < Re_k < 400$, where $Re_k$ is the Reynolds number based on roughness height $k$. They also present a numerical approach to model the effect of roughness on transition by adding a scalar field variable. The new scalar variable is used to modify the $\gamma-Re_{\theta}$  transition model \cite{menter2006transition}. Sareen et al. \cite{sareen2014effects} note that most of the experimental studies on roughness use strips or zigzag tapes to simulate real roughness and not many studies exist on negative roughness like erosion where material is lost from the blade. 

Apart from causing early transition, the nature of the turbulent boundary layer also changes due to roughness. Skin friction increases and a shift in the velocity profile in the inner part of the boundary layer is observed. The additional dissipation near the roughness elements leads to thickening of the boundary layer which can make the boundary layer prone to early separation. %In this paper, we explore different methods to model the effect of roughness in RANS simulations and the effects of roughness on the turbulent boundary layer. 

The concept of equivalent sand grain roughness is widely used in turbulence models to account for the effect of roughness on turbulent boundary layers. Nikuradse\citep{nikuradse1950laws} performed experiments to measure pressure losses across pipes due to roughness, which forms the basis of the sand grain roughness concept. Nikuradse provided relations for the loss in pressure head (friction) and the velocity shift as a function of sand grain roughness heights. Real roughness is first converted to equivalent sand grain roughness when using the roughness models for RANS turbulence models.
Typically the rough surface is replaced by a smooth surface and the effect of roughness is modeled as extra dissipation in the inner boundary layer.%The modifications to turbulence models to account for the effect of roughness are based on these equivalent sand grain roughness heights.  

Integral boundary layer based tools like RFOIL\cite{rfoil_orig} are used extensively in the wind energy community for quick and accurate analysis of airfoil performance, especially in combination with other methods like Blade Element Momentum theory, to obtain the power output of wind turbines in a relatively inexpensive manner. However, it is restricted mainly to clean airfoils due to lack of research on developing roughness models for integral boundary layer methods. Olsen et al\cite{olsen2020improved} recently proposed a new closure relation for skin friction in the presence of roughness. The authors also note that further work is necessary to refine their study. 

In this study, roughness models for the SA and SST $k-\omega$ turbulence models are implemented in the open source tool SU2\cite{SU22014}. The grid requirements and the accuracy of the two models are examined and validated against experimental data. Two airfoils are considered - NACA $65_2215$ and a popular wind turbine airfoil DU-96-W-180. The NACA $65_2215$ airfoil has been used for validating roughness models earlier\cite{hellsten1997extension}\cite{knopp2009new}. Sareen et al. \cite{sareen2014effects} performed experiments on the DU-W-96-180 with 'negative' roughness. Thus different ways to obtain equivalent sand grain roughness for 'negative' roughness are also examined in this paper. The numerical solution of the RANS equations is then used to analyze the behaviour of the turbulent boundary layer and the various integral boundary layer quantities in the presence of roughness as well as to analyze the integral boundary layer parameters in order to improve roughness modeling in integral boundary layer methods.

The organization of the paper is as follows: the two different roughness models for RANS are presented in section~\ref{sec:roughness}, validation cases for the roughness models are presented in section \ref{sec:fpvalid}. Based on the results in section \ref{sec:fpvalid}, the SA roughness model is validated against experiments on airfoils in section \ref{sec:airfoilappln}. In section \ref{sec:blanalysis}, the effect of roughness on various integral boundary layer properties is analysed. The conclusions are presented in section \ref{sec:conclusions}. 

